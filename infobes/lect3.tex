\documentclass[a4paper, 12pt]{article}

\usepackage[T2A]{fontenc}
\usepackage[utf8]{inputenc}
\usepackage[english,russian]{babel}

\begin{document}

\section{2.4 Гомоморфизмы}

Различные алгебры одной и той же сигнатуры носят название подобных

Понятие гомоморфизма:

$$
    f 
$$

$$
a -> f(a)
$$

$$
func_{i}(x, func_j(x, y))
$$

....

Пример: 

Сигнатура: $\Sigma_{int} = <S_{int}, \Omega_{int}>$

$S_{int} = {number}$

$\Omega_{int} = {zero, one: number; plus, minus: number, number -> number}$

Mod4 (см пред лекцию)

Mod2:

представление: $number = \{ 0, 1\}$

операции: $zero = 0, one = 1;$

(плюс и минус в табличках)

Гомоморфизм $f: |Mod4| -> |Mod2|$: 
$$
    f(0)=0 \\
    f(1) = 1 \\
    f(2) = 0 \\
    f(3) = 1 \\
$$

$$f(zero) |_{Mod4} = zero|_{Mod2}$$
$$f(one) |_{Mod4} = one|_{Mod2}$$
$$f(plus(a, b)) |_{Mod4} = plus(f(a), f(b))|_{Mod2}$$
$$f(times(a, b)) |_{Mod4} = times(f(a), f(b))|_{Mod2}$$

\underline{Формальное определение гомоморфизма:}

$$
\Sigma = <S, \Omega>
$$

Тогда $\Sigma$ - гомоморфизм из 
Есть такое семейство отображений f
$$ f_s(\omega(a_1, ... a_n)) = \omega(f_{u1}(a_1), ... f_{un}(a_n))  $$
$$ f_s(\omega(a_1, ... a_n)) = \omega(f_{u1}(a_1), ... f_{un}(a_n))  $$
$$ f_s(\omega(a_1, ... a_n)) = \omega(f_{u1}(a_1), ... f_{un}(a_n))  $$

что для всех элементов $$\omega \in \Omega_{us}, a_i \in |A|$$
$$ f_s(\omega(a_1, ... a_n)) = \omega(f_{u1}(a_1), ... f_{un}(a_n))  $$
$$ f_s(\omega(a_1, ... a_n)) = \omega(f_{u1}(a_1), ... f_{un}(a_n))  $$


\end{document}
